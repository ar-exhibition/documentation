\documentclass[12pt,a4paper]{article}
\usepackage[utf8]{inputenc}
\usepackage[german]{babel}
\usepackage[T1]{fontenc}
\usepackage{amsmath}
\usepackage{amsfonts}
\usepackage{amssymb}
\usepackage[left=2cm,right=2cm,top=2cm,bottom=2cm]{geometry}
\author{Marlon Lückert, Julius Neudecker, Vincent Schnoor}
\title{
Projekt 2 Dokumentation
}
\begin{document}

%1. - Beschreibung des Projektes, Ziel, Forschungsfragen, Zielgruppe
%
%2.1 - Technische Umsetzung Beschreibung, Skizzen, Bilder, …
%2.2- Recherche, Literaturübersicht, State of the Art Technik, Diskussion der existierenden Ansätze mit der eigenen Lösung und die innerhalb des Projektes zu meisternden Herausforderungen (W-Fragen)
%
%3.1 - User Research, gewählte Methoden, warum habt ihr die gewählt, wie habt ihr sie für eure Evaluation eingesetzt?
%3.2- Analyse Nutzerbefragung
%
%4. Schlussfolgerungen aus 2 und 3
%5. Ausblick für Weiterentwicklungen …
%6. Appendix in Form von Daten

\maketitle
\section{Projektbeschreibung}
Das Projekt \textit{ARExhibition}, welches im Rahmen des Projektes \textit{XRchitecture} entstanden ist macht das Ausstellen digitaler Modelle und Medien einfacher. Mit nur wenigen Klicks können 3D-Modelle, Bilder und Videos aus einem hierfür entwickelten Content-Management-System (CMS) in AR platziert und gespeichert werden. Die so entstandenen AR-Szenen können von Besuchern über einen Marker geladen und betrachtet werden.\\
Diese Ausstellungen können bspw. von Hochschulen und anderen Bildungsinstituten genutzt werden um Projekte der Studierenden und Schüler auszustellen.
\subsection{Projektziel}
Ziel des Projektes ist es, die Darstellung medialer Inhalte der Hochschule und anderer Lehrinstitute einfacher zu gestalten und eine Oberfläche für die Studenten der HAW zu schaffen um ihre im Semester erstellten 3D-Modelle, Bilder und Videos an einem zentralen Ort zu speichern und mit anderen Studierenden zu teilen.
\subsection{Zielgruppen}
Für unser aus zwei Teilen bestehendes Projekt gibt es zwei Zielgruppen.\\
Die erste Zielgruppe besteht aus den Studenten der HAW welche das Content-Management-System zum einfachen Teilen ihrer Arbeiten nutzen können. Die Projekte anderer Studiengänge und Fakultäten können so leichter eingesehen werden. Außerdem können die Arbeiten anderer Studenten heruntergeladen und für eigene Studienprojekte verwendet werden.\\

Bildungsinstitute wie die HAW Hamburg und Bildungsstätten wie Museen bilden die zweite Zielgruppe. Für bspw. die HAW Hamburg wird die Ausrichtung von Ausstellungen der Studentenprojekte vereinfacht, da Räume interaktiver und sinnvoller mit digitalen Medien gefüllt werden können. Wo vorher 3D-Objekte auf PC Bildschirmen betrachtet werden mussten, können diese mithilfe der App als Teil des Raumes betrachtet und mit ihnen interagiert werden. Museen und Ausstellungen können durch die App ihr Repertoire an Kunst und Objekten erweitern, da virtuelle Bilder und Objekte einfach ausgestellt werden können. Die Inhalte können genau wie vom Kurator gewollt platziert werden und erscheinen dem Besucher in der gewollten Position, Rotation und Größe.

\section{Technische Umsetzung}
Das Projekt besteht aus zwei gleich großen Bestandteilen: einem Content-Management-System, welches für die Bereitstellung des Inhalts der App und als Speichermedium studentischer Projekte dient und der App, welche die Platzierung virtueller Inhalte in einem Raum in AR ermöglicht.\\
Im Folgenden wird die technische Umsetzung des CMS und der App erläutert. Dabei wird auf bereits existierende ähnliche Produkte eingegangen und wie unser Projekt sich von diesen abgrenzt. Verschiedene Ansätze der Umsetzung werden diskutiert und miteinander verglichen und die Schwierigkeiten der Umsetzung werden erläutert.
\subsection{Stand der Forschung}
\subsection{Content-Management-System}
\subsubsection{Aufbau der Datenbank}
\subsection{AR App}
\section{Wissenschaftliche Umsetzung}
\subsection{Forschungsfrage}
\glqq LidAR besser? \grqq{}
\subsection{User Research}
\subsection{Analyse der Ergebnisse}
\section{Schlussfolgerungen}
\section{Ausblick}
\end{document}